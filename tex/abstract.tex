% Δεύτερη σελίδα: Περίληψη και λέξεις - κλειδιά (στην ελληνική και στην αγγλική)

\addcontentsline{toc}{chapter}{Περίληψη}
\chapter*{Περίληψη}

Η παρούσα εργασία εξετάζει τη δυνατότητα ενός δεδομένου αλγορίθμου χάραξης τριγώνου να γενικευθεί σε μία διαδικασία χάραξης αυθαίρετων κυρτών πολυγώνων, υπολογίζοντας αυξητικώς τις γραμμικές συναρτήσεις των ακμών τους για όλα τα εικονοστοιχεία του περιβάλλοντος κυτίου τους. Αρχικά, εξοικειωνόμαστε με τις έννοιες και τους όρους της υπολογιστικής γεωμετρίας, και πραγματοποιούμε μία εισαγωγή στον απλό αλγόριθμο ψηφιδόξυσης ενός πολυγώνου. Στη συνέχεια, μελετάμε το θεωρητικό υπόβαθρο της εργασίας, αναλύοντας τις μεθόδους ελέγχου εσωτερικού σημείου, καθώς και τους αλγόριθμους ψηφιδόξυσης πολυγώνων. Για την υλοποίηση της διαδικασίας, επιστρατεύουμε τον αλγόριθμο περιτυλίγματος με στόχο την εύρεση του κυρτού περιβλήματος του πολυγώνου και τον έλεγχο κυρτότητας αυτού, ενώ παράλληλα επισημαίνουμε τις ομοιότητες του δοθέντος αλγορίθμου χάραξης τριγώνων με τη γενικευμένη μορφή του ως αλγόριθμος χάραξης  πολυγώνων. Τέλος, με βάση τα συμπεράσματα, διαπιστώνουμε πως ορθώς υλοποιείται η διαδικασία χάραξης αυθαίρετων κυρτών πολυγώνων, και θέτουμε μελλοντικούς στόχους μελέτης και εφαρμογής των υπολοίπων αλγορίθμων που αναφέρθηκαν στο πλαίσιο της εργασίας, καθώς και βελτιστοποίησης του ήδη υλοποιημένου αλγορίθμου με τη χρήση του λογισμικού \textlatin{OpenGL}.

\vspace{1.5em}

\section*{Λέξεις - κλειδιά}
Κυρτά πολύγωνα, ψηφιδόξυση, κυρτό περίβλημα, αλγόριθμος περιτυλίγματος, \textlatin{OpenGL}

\newpage

\chapter*{\textlatin{Abstract}}

\textlatin{The present study examines the possibility of a given triangle rasterization algorithm being generalized to a process of rasterizing arbitrary convex polygons, incrementally calculating the linear functions of their edges for all pixels of their bounding box. First, we familiarize ourselves with the concepts and terms of computational geometry, and make an introduction to the simple algorithm of polygon rasterization. Next, we study the theoretical background of our study, by analyzing the interior-point methods, as well as the polygon rasterization algorithms. To implement the process, we use the gift-wrapping algorithm in order to find the convex hull of the polygon and check its convexity, while we also highlight the similarities between the given triangle rasterization algorithm and its generalized form as a polygon rasterization algorithm. Finally, based on the conclusions, we realize that the process of rasterizing arbitrary convex polygons is correctly implemented, and we set future goals for the study and application of the rest of the algorithms mentioned in this study, as well as the optimization of the already implemented algorithm using the OpenGL software.}

\vspace{1.5em}

\section*{\textlatin{Key Words}}
\textlatin{Convex polygons, rasterization, convex hull, gift-wrapping algorithm, OpenGL}
