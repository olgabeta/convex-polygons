% Συμπεράσματα

\addcontentsline{toc}{chapter}{Συμπεράσματα}
\chapter*{Συμπεράσματα}

Όπως έχει ήδη αναφερθεί, σκοπός της παρούσας εργασίας ήταν η γενίκευση και υλοποίηση της διαδικασίας \textlatin{triangle1} στην \textlatin{convex1}, η οποία χαράσσει αυθαίρετα κυρτά πολύγωνα, υπολογίζοντας αυξητικώς τις γραμμικές συναρτήσεις των ακμών τους για όλα τα εικονοστοιχεία του περιβάλλοντος κυτίου τους. Διαπιστώσαμε πως η γενίκευση της διαδικασίας \textlatin{triangle1} είναι εφικτή και δύναται να υλοποιηθεί με μία πληθώρα διαφορετικών μεθόδων, αναλόγως τις απαιτήσεις του χρήστη. Επιπρόσθετα, παρ’ όλο που η διαδικασία υλοποιήθηκε βάσει του αλγορίθμου περιτυλίγματος, αναγνωρίζουμε ότι η χρονική πολυπλοκότητα της χειρότερης περίπτωσης του εν λόγω αλγορίθμου τον καθιστά μη αποδοτικό για μεγάλο όγκο σημείων ενός κυρτού πολυγώνου.  \par

Δοθέντος αρκετού χρόνου για εμβάθυνση στο θέμα της εργασίας και εξοικείωση με την προγραμματιστική πλατφόρμα \textlatin{OpenGL}, στοχεύουμε στην εκτενέστερη εξέταση και την βελτιστοποίηση των υλοποιημένων αλγορίθμων ψηφιδόξυσης κυρτών πολυγώνων με τη χρήση του προαναφερθέντος λογισμικού. Στους μελλοντικούς μας στόχους συμπεριλαμβάνεται και η μελέτη των υπολοίπων αλγορίθμων χάραξης πολυγώνων που αναφέρθηκαν στο πλαίσιο της παρούσας εργασίας.
