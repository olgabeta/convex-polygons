% Αναφορές

\addcontentsline{toc}{chapter}{Αναφορές}

\begin{thebibliography}{16}

% Ελληνόγλωσση βιβλιογραφία

\bibitem{1} Δρακόπουλος, Β. (2021). \emph{Εισαγωγή και Αλγόριθμοι ψηφιδόξυσης} [Πανεπιστημιακές Σημειώσεις]. Πανεπιστήμιο Θεσσαλίας, Τμήμα Πληροφορικής με Εφαρμογές στη Βιοϊατρική, Π.Π.Σ.: "Γραφική Υπολογιστών", Εαρινό Εξάμηνο 2020-2021, Λαμία

\bibitem{2} Μουστάκας, Κ., Παλιόκας, Ι., Τζοβάρας, Δ., \& Τσακίρης, Α. (2015). \textit{Γραφικά και Εικονική Πραγματικότητα} [ηλεκτρ. βιβλ.]. Αθήνα: Σύνδεσμος Ελληνικών Ακαδημαϊκών Βιβλιοθηκών. \textlatin{\url{http://hdl.handle.net/11419/4491}}

\bibitem{3} Σπύρου, Ε. (2019). \emph{Αλγόριθμοι Σχεδίασης} [Πανεπιστημιακές Σημειώσεις]. Πανεπιστήμιο Θεσσαλίας, Τμήμα Πληροφορικής και Τηλεπικοινωνιών, Π.Π.Σ.: "Γραφικά", Εαρινό Εξάμηνο 2018-2019, Λαμία

\bibitem{4} Παληός, Λ. (2021). \emph{1. Διαίρεση Πολυγώνων σε Τρίγωνα, 2. Το πρόβλημα του Κυρτού Πολυγώνου στις Δύο Διαστάσεις} [Πανεπιστημιακές Σημειώσεις]. Πανεπιστήμιο Ιωαννίνων, Τμήμα Μηχανικών Η/Υ και Πληροφορικής, Π.Π.Σ.: "Υπολογιστική Γεωμετρία", Εαρινό Εξάμηνο 2020-2021, Ιωάννινα

\bibitem{5} Κωστόπουλος, Π. (2013). \emph{Μέθοδοι κατασκευής κυρτών περιβλημάτων και εφαρμογές}. [Μεταπτυχιακή Εργασία]. Αμητός: Αποθετήριο Πανεπιστημίου Πελοποννήσου. \textlatin{\url{https://amitos.library.uop.gr/xmlui/handle/123456789/980}}.

\bibitem{6} \textlatin{Hearn, D., \& Baker, M. P.} (2010). \emph{Γραφικά Υπολογιστών με \textlatin{OpenGL}} (Π. Μποζάνης, Επιμ.) (Γ. Σίσιας, Μετάφ.) (3η έκδ.). Θεσσαλονίκη: ΤΖΙΟΛΑ. (Το πρωτότυπο έργο δημοσιεύτηκε το 1986).

\bibitem{7} Δρακόπουλος, Β. (2022). \emph{Εισαγωγή στην \textlatin{OpenGL}} [Εργαστηριακές Σημειώσεις]. Πανεπιστήμιο Θεσσαλίας, Τμήμα Πληροφορικής με Εφαρμογές στη Βιοϊατρική, Π.Π.Σ.: "Γραφική Υπολογιστών", Εαρινό Εξάμηνο 2021-2022, Λαμία

\bibitem{8} Εργαστήριο Τεχνολογίας Πολυμέσων (\textlatin{medialab}) (2017). \emph{Κεφάλαιο 1: Βασικές αρχές σχεδίασης} [Πανεπιστημιακές Σημειώσεις]. Εθνικό Μετσόβιο Πολυτεχνείο, Τμήμα Ηλεκτρολόγων Μηχανικών και Μηχανικών Υπολογιστών, Αθήνα. Ανακτήθηκε από: \textlatin{\url{ http://www.medialab.ntua.gr/education/ComputerGraphics/OpenGL_Lectures/02-Chapter1.pdf}}

\vspace{2.5em}

% Ξενόγλωσση βιβλιογραφία

\bibitem{9} \textlatin{Kjeldsen, T. H. (2010). History of convexity and mathematical programming: connections and relationships in two episodes of research in pure and applied mathematics of the 20th century. In R. Bhatia (Ed.), \emph{Proceedings of the International Congress of Mathematicians 2010 - Invited lectures}, 4, 3233-3257. Hindustan Book Agency.}

\bibitem{10} \textlatin{Graham, R.L. (1972). An efficient algorithm for determining the convex hull of a finite planar set. \emph{Information Processing Letters}, 1(4), 132–133. doi: 10.1016/0020-0190(72)90045-2.}

\bibitem{11} \textlatin{Jarvis, R. A. (1973). On the identification of the convex hull of a finite set of points in the plane. \emph{Information Processing Letters}, 2(1), 18-21. doi: 10.1016/0020-0190(73)90020-3.}

\bibitem{12} \textlatin{Chan, T.M. (1996). Optimal output-sensitive convex hull algorithms in two and three dimensions. \emph{Discrete \& Computational Geometry}, 16, 361–368. doi: 10.1007/BF02712873.}

\bibitem{13} \textlatin{Sharma, P. (2018, July 2). Gift Wrap Algorithm (Jarvis March Algorithm) to find Convex Hull. \emph{OpenGenus IQ: Learn Computer Science}}. Ανακτήθηκε από: \textlatin{\url{https://iq.opengenus.org/gift-wrap-jarvis-march-algorithm-convex-hull/}}

\bibitem{14} \textlatin{Neider, J., \& Davis, T. (1997). \emph{Opengl Programming Guide: The Official Guide to Learning Opengl, Version 1.1, Second Edition}. Addison-Wesley Publishing Company.} Ανακτήθηκε από: \textlatin{\url{http://www.cse.chalmers.se/edu/course/TDA362/redbook.pdf}}

\bibitem{15} \textlatin{Wang, Y. F. (2017). \emph{OpenGL Input}} [Πανεπιστημιακές Σημειώσεις]. \textlatin{University of California, Computer Science Department}, Π.Π.Σ.: \textlatin{"Introduction to Computer Graphics"}, Χειμερινό Εξάμηνο 2017-2018, \textlatin{Santa Barbara}

\bibitem{16} \textlatin{Agu, E. O. (2016). \emph{Lecture 3: Introduction to OpenGL/GLUT (Part 2)}} [Πανεπιστημιακές Σημειώσεις]. \textlatin{Worcester Polytechnic Institute (WPI), Computer Science Department}, Π.Π.Σ.: \textlatin{"Computer Graphics"}, Εαρινό Εξάμηνο 2016, \textlatin{Worcester}

\end{thebibliography}
